Le \red{PPCM} signifie \red{P}lus \red{P}etit \red{C}ommun \red{M}ultiple.

\begin{definition}
Soient a et b deux entiers. Les assertions suivantes sont équivalentes :
\begin{itemize}
    \item a est un multiple de b
    \item b est un diviseur de a
    \item b|a
    \item $\exists k \in \ens{Z} \ \text{tel que} \ bk = a$
\end{itemize}
\end{definition}
Le PPCM de deux entiers a et b, noté $a \vee b$, est donc le plus petit entier naturel qui soit à la fois un multiple de a et de b.

\begin{example}
$3 \vee 5 = 15$ ; $2 \vee 6 = 6$
\end{example}

Bien que moins étudié que le PGCD, le PPCM possède néanmoins lui aussi quelques propriétés intéressantes.

La plus notable est la suivante :

\begin{theorem}
Soient $m = a \vee b$, et q un entier. On a alors :\\
$(a|q)$ et $(b|q) \iff (m|q)$
\end{theorem}